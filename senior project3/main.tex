\documentclass[10pt,twocolumn]{article}

% use the oxycomps style file
\usepackage{oxycomps}
\usepackage{listings}
\usepackage{xcolor}
\usepackage{geometry}
\usepackage{graphicx}
\usepackage{caption}

% Define custom colors for code listings
\definecolor{codegreen}{rgb}{0,0.6,0}
\definecolor{codegray}{rgb}{0.5,0.5,0.5}
\definecolor{codepurple}{rgb}{0.58,0,0.82}
\definecolor{backcolour}{rgb}{0.95,0.95,0.92}

% Define code listing style
\lstdefinestyle{mystyle}{
    backgroundcolor=\color{backcolour},
    commentstyle=\color{codegreen},
    keywordstyle=\color{magenta},
    numberstyle=\tiny\color{codegray},
    stringstyle=\color{codepurple},
    basicstyle=\footnotesize\ttfamily,
    breakatwhitespace=false,
    breaklines=true,
    captionpos=b,
    keepspaces=true,
    numbers=left,
    numbersep=5pt,
    showspaces=false,
    showstringspaces=false,
    showtabs=false,
    tabsize=2
}

% usage: \fixme[comments describing issue]{text to be fixed}
% define \fixme as not doing anything special
\newcommand{\fixme}[2][]{#2}
% overwrite it so it shows up as red
\renewcommand{\fixme}[2][]{\textcolor{red}{#2}}
% overwrite it again so related text shows as footnotes
%\renewcommand{\fixme}[2][]{\textcolor{red}{#2\footnote{#1}}}

% read references.bib for the bibtex data
\bibliography{references}

% include metadata in the generated pdf file
\pdfinfo{
    /Title (Random Table-top Game Map Generator)
    /Author (Victor Zhu)
}

% set the title and author information
\title{Random Table-top Game Map Generator}
\author{Victor Zhu}
\affiliation{Occidental College}
\email{hzhu@oxy.edu}

\begin{document}

\maketitle



\section{Introduction}

In the community of tabletop role-playing games, DM(Dungeon Master) or Game Runner can be a very tedious job. They must create the game, the adversary, the obstacles, and the reward. So, a random map generator that can randomly create a map with monsters, loots, and NPCs would be really helpful. One of my scripts will create a randomized map, and Another generator will generate a random boss room layout. These will make both DM and players' gaming experience much easier than before.    

Creating immersive and engaging game environments has long been a central pursuit for me. A fundamental aspect of such environments is the intricately designed maps that provide players with a sense of place, adventure, and mystery. A sophisticated and innovative tool - a Random Dungeon and Boss Map Generator- is developed in this pursuit.

I have developed a Random Dungeon and Boss Map generator: the Dungeon Map and the Boss's Map. These maps result from two separate algorithms, each playing a crucial role in enhancing the gaming experience.
The first algorithm I created, Dungeon Map Generator, was inspired by the algorithm "Drunkyard's Walk." The drunkard in the Drunkyard's namesake. takes inspiration from the unpredictable movements of an intoxicated individual. This algorithm introduces an element of randomness and unpredictability to the generated maps, adding an exciting element of surprise for players. My algorithm is different than the generic Drunkyard's walk. I will explain it in more detail in the method.
The second algorithm I designed is a random boss room generator- it gives more constraints than the random dungeon map generator. These algorithms ensure the maps are rich in detail and logic, providing players with immersive and engaging environments to explore.

\section{Problem Context}

Tabletop role-playing games (RPGs) have long captivated players' imaginations, offering rich narratives and immersive experiences. At the heart of these games lies the Dungeon Master (DM) or Game Runner, a dedicated individual responsible for crafting the game world, guiding players through adventures, and orchestrating the story. However, one undeniable challenge that DMs face is the extensive time commitment required for preparation before each gaming session. On average, DMs dedicate 3 to 10 hours per week\cite{quora_dnd_duration} \cite{reddit_dnd_prep}to prepare for a single gaming session, a substantial investment that can often deter individuals from taking on this pivotal role.
Recognizing the significance of this issue, I have started a project aimed at alleviating the preparation burden borne by DMs. The core objective of this project is to provide a practical and efficient tool that streamlines the preparation process, thereby enabling DMs to dedicate more time to enjoying the game and fostering engaging experiences for players.
This project's societal context is integral to understanding its value and impact. Tabletop RPGs serve as a source of entertainment and facilitate social interaction, creativity, and critical thinking. In a world where digital screens often dominate leisure time, tabletop RPGs offer a unique and valuable analog experience that fosters face-to-face communication and collaboration. By reducing the preparation time for DMs to less than 30 minutes, this project seeks to lower barriers to entry, making it more accessible for individuals to take on the role of a DM. The project promotes the growth of tabletop gaming communities, enriching social connections and encouraging creative storytelling.
Moreover, the benefits of this project extend beyond leisure and recreation. The project aligns with the broader trend of leveraging technology to enhance traditional gaming experiences. It reflects a broader societal shift towards innovation, efficiency, and accessibility, aligning with the digital age's demands for streamlined processes and user-friendly tools.
 In a broader societal context, it aligns with the increasing integration of technology into leisure activities, catering to modern society's evolving needs and expectations. Ultimately, this project stands to make a valuable contribution to both the gaming community and the broader social landscape, where the pursuit of shared experiences and creative expression is highly prized.


\section{Technical Background}

In this section, I will briefly touch on the technical background of the code for generating DND boss room layouts and introduce Drunkard's Walk algorithm for context. The Random Dungeon Map Generator code operates based on predefined room types and constraints, adhering to placement rules and validating room assignments. It ensures a balanced and thematic layout within a grid-based environment, enhancing tabletop role-playing game experiences. On the other hand, Drunkard's Walk algorithm is a more straightforward yet versatile method for generating random patterns within grids, widely applicable in various procedural generation tasks. These technical insights provide a foundation for understanding the code's functionality and relationship to established algorithms in procedural generation.
 
The code for Random Boss Map Generator code generates a "boss" room layout on a grid. It defines various room types with maximum counts, follows the rules for room placement (e.g., Throne Room far from Entrance), checks for valid placements based on criteria, and fills the grid with rooms. Matplotlib is used for visualization, displaying each room type in distinct colors. This code streamlines the creation of thematic and balanced boss room layouts for tabletop RPGs.

The Drunkard's Walk algorithm is a straightforward yet effective method for generating random paths or patterns within a grid or space. Starting from an initial point, it repeatedly takes random steps in various directions, recording the visited locations to create meandering or branching paths. Widely used in procedural generation tasks, it can produce diverse and unpredictable outcomes, making it valuable for generating mazes, cave systems, or natural features like rivers and roads. The algorithm's simplicity and adaptability make it popular for creating intricate and randomized structures in various applications.


\section{Prior Work}

The methods of Creating Procedurally Generated Dungeons are various and diverse, each contributing to the dynamic landscapes of digital and tabletop gaming. This section delves into the historical context and evolution of procedural dungeon generation, outlining key methodologies while positioning the current project within this broader spectrum of prior work.

The genesis of procedurally generated dungeons can be traced to the domain of tabletop role-playing games (RPGs), with "Dungeons & Dragons" being a seminal influence. Early RPGs utilized random tables to determine dungeon layouts, room contents, and encounters, imbuing each gameplay session with a sense of unpredictability and adventure. This concept of procedural generation was later adopted and refined within the realm of video games, marking a significant evolution from its tabletop origins. A landmark in this journey was the game "Rogue" (1980), which introduced players to endlessly unique dungeon experiences, setting a precedent for the use of procedural content generation (PCG) in game design.\cite{Holmes2020Rogue}
\begin{figure}
    \centering
    \includegraphics[width=0.5\linewidth]{rougue.PNG}
    \caption{Rougue}
    \label{fig:enter-label}
\end{figure}

As the application of PCG expanded, a variety of algorithms emerged, each suited to different types of games and experiences:

Binary Space Partitioning (BSP): Employed in games like "Doom," BSP generates structured, non-overlapping rooms and corridors, ideal for creating man-made structures.\cite{TwoBitHistory2019DoomBSP}

Cellular Automata: This method, utilized in "Minecraft" for cave systems, excels at creating naturalistic, organic environments through the simulation of cellular evolution.\cite{HeardProceduralDungeonCA}

Inspired by the detailed yet constrained dungeons of Watabou's Procgen Arcana and the procedural generation insights from a ClojureScript blog post on the Drunkard’s Walk algorithm, this project sought a methodology that balanced simplicity with the capacity to generate intricate, engaging dungeons. The project adopts a modified version of the Drunkard’s Walk algorithm, distinguished by its four-step simplification, which contrasts with the more common yet complex algorithms like BSP or Cellular Automata.
\begin{figure}
    \centering
    \includegraphics[width=0.5\linewidth]{watab.png}
    \caption{Watabou's Procgen Arcana}
    \label{fig:enter-label}
\end{figure}

The choice to adapt the Drunkard’s Walk algorithm was influenced by several factors:

- **Simplicity and Flexibility**: The algorithm’s straightforward nature facilitates ease of implementation and experimentation, allowing for the generation of unique dungeon layouts with minimal complexity.

Balance of Randomness and Structure: Unlike the highly structured outcomes of BSP or the organic randomness of Cellular Automata, the modified Drunkard’s Walk algorithm strikes a balance, creating dungeons that are both unpredictable and navigable.

Adaptability: This algorithm's adaptable framework supports diverse game types and narratives, enabling the creation of customized dungeon experiences that align with specific design goals.

The exploration and selection of procedural generation methodologies for dungeons are informed by a rich history of prior work across both tabletop and video gaming. By situating this project within the context of existing methodologies, the rationale behind adopting a simplified Drunkard’s Walk algorithm becomes clear. This choice not only acknowledges the contributions of prior work but also highlights the project's unique approach to generating procedurally created dungeons, underscoring the ongoing evolution and diversity in the field of procedural content generation.

I used the Procedural Dungeon Generation: A Drunkard's Walk in ClojureScript as inspiration for my early version of code. The blog post explores procedural dungeon generation using the "Drunkard's Walk" algorithm implemented in ClojureScript. The author discusses the algorithm's basics, where it starts from a point and randomly navigates the grid to create intricate maze-like structures. \cite{jrheard_dungeon_clojurescript}

\begin{figure}
    \centering
    \includegraphics[width=0.5\linewidth]{mine.png}
    \caption{my output from inspiration from Procedural Dungeon Generation: A Drunkard's Walk in ClojureScript}
    \label{fig:enter-label}
\end{figure}


The post provides insights into the implementation details, including grid initialization, wall generation, and room placement. It also highlights how parameters like randomness and wall density affect the generated dungeons. The blog post offers a practical demonstration of the algorithm's capabilities, showcasing the creation of diverse and engaging dungeon layouts for game development or other applications. However, one of the project's shortcomings is it doesn't guarantee a map for adventure. Sometimes, it looks like a giant opening instead of a dungeon pattern.
Then I found this webpage. It simplifies the drunkard walk algorithm into four steps:
\begin{enumerate}
    \item Pick a random point on a filled grid and mark it empty.
    \item Choose a random cardinal direction (N, E, S, W).
    \item Move in that direction, and mark it empty unless it already was.
    \item Repeat steps 2-3 until you have emptied as many grids as desired.\cite{pcgwiki_drunkardwalk}
\end{enumerate}

It is easy to follow, so I created my algorithm based on this blog.


\section{Methods}


The methodology for generating dungeon maps and boss room maps is informed by a series of discussions with experienced Dungeon Masters (DMs). These discussions revealed critical aspects that guided the development of two separate procedural generation programs: one for dungeons and another for boss rooms. This section outlines the key insights from DMs and how they shaped the project requirements and algorithmic constraints.

\subsection{Separate Programs for Dungeon and Boss Rooms}

\textbf{Narrative Depth:} DMs highlighted the narrative significance of boss rooms as climactic points in gameplay, requiring detailed attention to enhance storytelling opportunities. This feedback led to the development of a dedicated boss room generator, focusing on creating immersive, narrative-rich encounter spaces.

\textbf{Complexity and Detail:} Given the pivotal role of boss rooms in gameplay, DMs recommended more complex and detailed designs than standard dungeon rooms. This necessitated a specialized generator capable of handling unique design constraints to craft memorable boss encounters.

\textbf{Customization Needs:} DMs expressed a need for adaptable thematic elements to align with varying campaign narratives. A separate boss room generator allows for this level of customization, catering to specific DM preferences and enhancing thematic coherence.

\subsection{Algorithm Constraints for Enemy and Treasure Generation}

\textbf{Balanced Encounters:} The importance of gameplay balance was a recurrent theme in DM feedback. The project incorporated algorithmic constraints to ensure encounters are challenging yet fair, adjusting enemy difficulty in relation to player levels and the dungeon's progression.

\textbf{Narrative and Gameplay Integration:} DMs stressed that the placement of treasures and enemies should serve the narrative and gameplay. This led to the implementation of constraints ensuring that loot rarity and enemy types contribute meaningfully to the story and the challenge of each room.

\textbf{Player Reward System:} Acknowledging the need for rewarding exploration and achievements, the project devised algorithms to place treasures strategically, ensuring rewards are commensurate with the challenges encountered.

\subection{Implementation Details}

The methodology combines insights from DMs with procedural generation principles to create engaging and diverse dungeons and boss rooms. The \textit{Random Dungeon Map Generator} employs varied room shapes and meticulous layout planning to prevent overlap, ensuring a coherent and exploratory experience. Strategic connectivity and room labeling add layers of narrative depth, guiding player progression and enriching the storytelling.

\begin{figure}[h]
\centering
\includegraphics[width=0.5\linewidth]{mymap.png}
\caption{Example of a Random Dungeon Map}
\label{fig:dungeon-map}
\end{figure}

The \textit{Random Boss Room Generator} focuses on spatial hierarchy and thematic placement, influenced by DMs' emphasis on the narrative significance of boss encounters. It ensures that crucial rooms are placed logically within the dungeon to build anticipation and challenge.

\begin{figure}[h]
\centering
\includegraphics[width=0.5\linewidth]{bossmap.png}
\caption{Example of a Boss Room Layout}
\label{fig:boss-room}
\end{figure}

This methodology, enriched by direct DM feedback, establishes a framework for generating dungeons and boss rooms that balance randomness with strategic design, narrative depth, and gameplay balance. By adhering to logical constraints and incorporating DM insights, the project delivers well-designed and engaging dungeon layouts that elevate the Dungeons and Dragons (DnD) gaming experience.

\section{Evaluation Metrics}

These are the evaluation metrics I created on the first day I designed the project.I think these are fitting for the project.

\begin{enumerate}
    \item \textbf{Good Storyline (Who What Why):}
    
    \textit{Importance:} A good storyline is crucial for engaging players in the game world. It provides context, purpose, and motivation for their actions. A well-crafted narrative can make the gameplay more immersive and enjoyable.\cite{reddit_dnd_campaign_fun}
    
    \item \textbf{Good Adversary (Motivation, Theme, Difficulty, Variation):}
    
    \textit{Importance:} The adversary or antagonist in a game adds challenge and conflict, making the gameplay interesting. Their motivation and theme contribute to the depth of the story. The difficulty and variation ensure that encounters with the adversary remain engaging and not repetitive.\cite{reddit_dnd_campaign_fun}
    
    \item \textbf{Good NPC (Background, Personality, Function):}
    
    \textit{Importance:} Non-player characters (NPCs) play a significant role in shaping the game world. Their backgrounds and personalities add richness to the narrative. Their functions can range from providing information to offering quests or trading, enhancing the player's experience.\cite{reddit_dnd_campaign_fun}
    
    \item \textbf{Good Resolution (Rewarding or not, Meaning):}
    
    \textit{Importance:} The resolution of the hero's journey is what players work toward. A rewarding resolution provides a sense of accomplishment and satisfaction, motivating players to continue or replay the game. The meaning or impact of the resolution can leave a lasting impression on players.\cite{reddit_dmacademy_oneshot}
    
    \item \textbf{Well-Constructed (What will the map look like, Theme, Difficulty):}
    
    \textit{Importance:} A well-constructed map is essential for navigation and exploration within the game. The map's theme sets the tone and atmosphere, enhancing immersion. Difficulty ensures that players are appropriately challenged, maintaining their interest.\cite{reddit_dmacademy_oneshot}
\end{enumerate}

A rewarding resolution provides a sense of achievement and closure, motivating players to complete the game and potentially play it again.
The importance of these metrics is justified in the context of game design literature, where a compelling narrative, challenging gameplay, and immersive world-building are often considered essential elements for successful games. These metrics help ensure that your algorithm generates game scenarios that are enjoyable, engaging, and memorable for players.



\section{Evaluation Results and Discussion}

\subsection{Methodological Approach}

My methodological approach involved the development of algorithms for procedural dungeon and boss room generation, incorporating feedback from Dungeon Masters (DMs) to ensure the created environments meet gameplay and narrative standards. Initial interviews with five DMs via Discord provided foundational insights that shaped algorithm development. Key areas of focus included enhancing narrative depth, ensuring logical room connectivity, and maintaining a balance of challenge and exploration within the generated maps.

\subsection{Evaluation Metrics}

To assess the effectiveness of my procedural generation algorithms, I established the following evaluation metrics based on DM feedback and game design literature:

\begin{enumerate}
    \item \textbf{Narrative Engagement:} Measured through DM and player feedback on the storyline's integration and immersive quality.
    \item \textbf{Adversary Design:} Evaluated by the variety and thematic alignment of challenges, as reported by playtest participants.
    \item \textbf{NPC Development:} Assessed through observations of player-NPC interactions and the NPCs' impact on game progression.
    \item \textbf{Map Construction Quality:} Determined by examining the logical arrangement of rooms and ease of navigation within the generated environments.
\end{enumerate}

\subsection{Evaluation Process and Results}

\textbf{Pre-Playtesting Interviews:} Prior to playtesting, interviews with DMs highlighted concerns regarding content depth and connectivity within the maps. They have point out that I need a more complex map and create an  of monster and traps. Adjustments were made accordingly, focusing on room descriptions and connectivity logic.

\textbf{Playtesting Sessions:} Engaging student DMs and players in playtesting sessions revealed the maps' immersive quality and narrative engagement. Feedback emphasized the need for more logical room arrangements and enhanced narrative elements.

\textbf{Feedback Integration:} Based on playtesting feedback, further refinements were made to address room arrangement consistency and the integration of narrative elements. The adjustments aimed at improving gameplay strategy and narrative depth.

\textbf{Quantitative Results:} Although specific quantitative metrics such as player engagement time and NPC interaction rates were not initially established, qualitative feedback suggested high levels of engagement and satisfaction with the narrative and gameplay elements.


\section{Ethical Considerations and Implementation}

\subsection{Fairness and Inclusivity}
\begin{itemize}
    \item my algorithms incorporate \textbf{fair randomization} techniques to ensure that the generated dungeons do not favor or disadvantage any player. This was achieved by using balanced randomization functions and regularly testing outcomes for bias.
    \item I focused on \textbf{accessibility} by designing user interfaces that are compatible with screen readers and providing descriptions in text format to aid visually impaired players.
    \item To promote \textbf{inclusivity}, the content generation algorithms were designed to avoid stereotypes and ensure a wide representation of characters and scenarios, fostering a positive gaming environment for all.
\end{itemize}

\subsection{Player Agency}
\begin{itemize}
    \item \textbf{Transparency} about the procedural generation process is provided through documentation and in-game explanations, giving players and DMs insight into how dungeons are created.
    \item I have enabled \textbf{customization} options allowing DMs to modify algorithm parameters to suit their campaign needs, enhancing player agency in shaping their adventure.
\end{itemize}

\subsection{Intellectual Property}
\begin{itemize}
    \item To respect \textbf{intellectual property}, all third-party assets are used with permission or are properly attributed, and I encourage original content creation within the TTRPG community.
\end{itemize}

\subsection{Data Privacy}
\begin{itemize}
    \item \textbf{Data protection} measures are implemented to ensure that any collected data, such as player preferences or custom dungeon settings, are securely stored and handled in compliance with data protection laws.
\end{itemize}

\subsection{Accountability and Continuous Improvement}
\begin{itemize}
    \item I am committed to \textbf{accountability}, with mechanisms in place for reporting and addressing biases, bugs, or ethical concerns raised by the community.
    \item I engage in \textbf{continuous improvement} by updating I algorithms based on player and DM feedback and staying informed about ethical standards within the gaming community.
\end{itemize}

\section{Conclusion}
Through these measures, my project aims to deliver a procedurally generated dungeon and boss room experience that is ethically responsible, inclusive, and enjoyable for the TTRPG community. I believe that considering these ethical aspects is crucial for the sustainable and positive development of gaming technologies.


\section{Conclusion}

In conclusion, the project of developing procedural dungeon and boss room generators for tabletop role-playing games (TTRPGs) has yielded promising results, addressing several critical aspects of game design and enhancing the overall player experience. I have focused on creating immersive, balanced, and ethical gaming environments throughout the project. As I reflect on the achievements and challenges encountered during this endeavor, it is clear that procedural generation has the potential to impact TTRPGs significantly and positively.

Key Achievements:
1. **Enhanced Gameplay**: The procedural dungeon generator has successfully created diverse and strategically structured dungeon layouts, offering players a more engaging and challenging experience. Including dynamic room, captions has added depth to storytelling, while the boss room generator has improved narrative coherence and tactical gameplay.

2. **Fairness and Inclusivity**: Ethical considerations have been central to my project, ensuring that generated content is fair, inclusive, and appropriate for diverse audiences. This approach promotes a welcoming and respectful gaming environment.

3. **Player Agency**: By allowing players and Dungeon Masters (DMs) to customize algorithms and adjust parameters,  I have empowered them to tailor their TTRPG adventures to their preferences. Transparency in procedural generation further enhances player agency, fostering a sense of control and immersion.

4. **Content Appropriateness**: The project has prioritized age-appropriate content, trigger warnings, and cultural sensitivity, making TTRPGs accessible and enjoyable for many players. This approach helps prevent potentially uncomfortable or distressing experiences. All the players and the DMs are comfortable with the content I created.

5. **Ethical Considerations**: I have proactively addressed issues related to intellectual property, data privacy, and accountability, ensuring that the project aligns with ethical standards and legal requirements to my players and DMs.
\section{Future Work}
While the project has achieved several milestones, there is ample room for further development and improvement in the field of procedural dungeon and boss room generation for TTRPGs:

1. **Advanced Algorithms**: Future work can explore more sophisticated algorithms to create even more complex and dynamic dungeon layouts. Incorporating decision trees, machine learning, or neural networks may enable generators to adapt to player preferences and behaviors during gameplay.

2. **Multiplayer Integration**: Integrating multiplayer capabilities, allowing multiple players or DMs to collaboratively or competitively create and explore procedurally generated dungeons, can enhance the social aspect of TTRPGs.

3. **Enhanced Graphics and Visualization**: Improving the graphical representation of generated dungeons, potentially through the integration of graphic design software or 3D modeling, can provide a more immersive visual experience.

4. **Natural Language Processing**: Incorporating natural language processing to generate detailed descriptions of rooms, NPCs, and encounters could further enrich storytelling and reduce the need for manually entering room descriptions.

5. **Player Feedback Mechanisms**: Developing systems for players to provide feedback on generated content and gameplay experiences can help fine-tune algorithms and identify areas for improvement.

6. **Expansion into Other TTRPG Systems**: Adapting the procedural generation tools to work seamlessly with various TTRPG systems beyond Dungeons & Dragons, such as Pathfinder or Call of Cthulhu, would broaden the project's applicability.

7. **Educational Applications**: Exploring how procedural generation can be used for educational purposes, such as teaching game design or history through TTRPG scenarios, can open up new avenues for the project.

8. **Dynamic Difficulty Scaling**: Implementing mechanisms that dynamically adjust dungeon difficulty based on player skill, character level, or party composition can create more tailored and challenging experiences.

9. **Storyline Integration**: Integrating procedural generation into overarching campaign storylines, allowing generated content to align more closely with the narrative, can enhance immersion and coherence.

10. **Cross-Platform Compatibility**: Ensuring that the procedural generation tools are compatible with various platforms, including virtual tabletop systems and mobile devices, can broaden their accessibility.

In summary, the project has made significant strides in developing procedural dungeon and boss room generators for TTRPGs, emphasizing fairness, inclusivity, and ethical considerations. The potential for growth and innovation in this field is substantial, offering exciting opportunities to enhance TTRPG experiences for players and DMs. By continuing to refine algorithms, expand capabilities, and embrace emerging technologies, I can look forward to a future where procedurally generated content plays an even more significant role in tabletop role-playing games.


\section{Timeline}

I commenced this project in the second week of the semester, and as I prepared for my initial presentation, I was uncertain about the path to success. In early October, a significant breakthrough occurred when I developed a program based on the Drunkard's Walk algorithm. However, I found this approach to be overly simplistic. Subsequently, in late October, I completed the prototype for my current project. Following a month of coding and refining, I conducted playtesting at the end of November. Additionally, I successfully created a boss room generator, which involved more intricate constraints than the dungeon map generator. Currently, I am in the process of composing this essay

\section{Replication Instructions}
\section{Prerequisites}

Make sure you have the following prerequisites installed on your system:

\begin{itemize}
    \item Python (3.x recommended)
    \item Matplotlib
    \item NumPy
\end{itemize}

You can install Matplotlib and NumPy using pip:

\begin{lstlisting}[language=bash]
pip install matplotlib numpy
\end{lstlisting}
\section{Tutoiral for Dungeonmap.py}
\subsection{Step 1: Clone the Repository (If Applicable)}

If your code is part of a Git repository, clone it to your local machine:

\begin{lstlisting}[language=bash]
git clone <https://github.com/VictorZhudd/senior-comp2.git>
cd <repository_directory>
\end{lstlisting}

\subsection{Step 2: Open the Python Script and make sure bossroom.py is in the folder}

Open the Python script where you've pasted the provided code. Let's call it \texttt{dungeonmap.py}. Since the code will create not only a dungeon map but also a boss room map, the user must find a function called generate\_dnd\_boss\_room\_layout\_v8 in another script. In my package, you can find it in bossroom.py

\subsection{Step 3: Adjust Room Type Chances}

You can customize the chance of different room types appearing by modifying the \texttt{group\_params} list. Each element in this list represents a group of rooms with associated labels, shapes, and chances.

For example, if you want to increase the chance of "Treasure" rooms appearing, change the chance value (between 0.0 and 1.0) in the \texttt{group\_params} list. Here's how:

\begin{lstlisting}[language=Python]
group_params = [
    (['Treasure', 'Puzzle'], ['rectangle', 'rectangle'], [0.8, 1.0]),  # Increase chance for 'Treasure'
    (['Treasure', 'Strong Monsters'], ['rectangle', 'circle'], [0.2, 0.6]),
    # Add or modify other room groups as needed
]
\end{lstlisting}

\subsection{Step 4: Define Custom Colors}

To define custom colors for room visualization, you can modify the \texttt{colors} array and the \texttt{cmap} colormap. The \texttt{colors} array represents the RGB colors for different room types, and the \texttt{cmap} is the colormap that uses these colors.

Here's how to define custom colors:

\begin{lstlisting}[language=Python]
colors = np.array([
    [0, 0, 0],     # Black for walls (-1)
    [0.8, 0.8, 0.2], # Custom color for 'Treasure' rooms (0)
    [1, 1, 1],     # White for paths (1)
])
cmap = ListedColormap(colors)
\end{lstlisting}

You can adjust the RGB values to specify different colors for each room type.

\subsection{Step 5: Customize Room Descriptions}

Customize room descriptions by modifying the \texttt{room\_descriptions} dictionary. Add or modify descriptions for each room type as needed.

Here's an example of custom room descriptions:

\begin{lstlisting}[language=Python]
room_descriptions = {
    'Treasure': [
        'A glittering treasure chest awaits! It contains a valuable gem.',
        'You found a hidden stash of gold coins and jewelry!',
    ],
    'NPC': [
        'You encounter a friendly merchant who offers to trade with you.',
        'A mysterious traveler shares valuable information about the dungeon.',
    ],
    # Add or modify descriptions for other room types
}
\end{lstlisting}

\subsection{Step 6: Specify Room Dimensions and Map Size}

You can specify the dimensions of rooms and the size of the grid by modifying relevant parameters in the code. For example, to change the minimum and maximum room dimensions, modify the following lines:

\begin{lstlisting}[language=Python]
room_width = random.randint(3, 6)  # Change the range as needed
room_height = random.randint(3, 6)  # Change the range as needed
\end{lstlisting}

Also, define the dimensions of the grid by changing the \texttt{width} and \texttt{height} variables:

\begin{lstlisting}[language=Python]
width, height = 30, 30  # Change to your desired map size
\end{lstlisting}

\section{Step 7: Run the Script}

Save your changes and run the script:

\begin{lstlisting}[language=bash]
python dungeonmap.py
\end{lstlisting}
and it should come out likes this:
\begin{figure}
    \centering
    \includegraphics[width=0.5\linewidth]{example1.png}
    \caption{the map part of you result should look like this}
    \label{fig:enter-label}
\end{figure}

\begin{figure}
    \centering
    \includegraphics[width=0.5\linewidth]{decs.png}
    \caption{the map should come with these descriptions}
    \label{fig:enter-label}
\end{figure}
it will also trigger bossroom.py and create a boss room map
\section{Tutorial for bossroom.py}
This tutorial provides an interactive guide to enhancing your DnD Boss Room Layout script. It will walk you through the features of the script and demonstrate how to effectively use and modify them to create compelling dungeon designs. This script connects to the dungeonmap.py. It needs both scripts to work.

\subsection{Exploring the Script's Features}
\subsubsection{Creating a Diverse Array of Rooms}
The heart of the script lies in its ability to create various room types. This diversity is managed through a dictionary named \texttt{room\_types}. Here's how to use it:

\begin{itemize}
    \item \textbf{Adding a New Room Type:} To introduce a new room, such as a 'Mystical Study', simply add it to the dictionary with a maximum count. For example:
    \begin{lstlisting}[language=Python]
    room_types["Mystical Study"] = 1
    \end{lstlisting}

    \item \textbf{Changing Room Names:} Modify the keys in the \texttt{room\_types} dictionary to rename rooms.
\end{itemize}

\subsubsection{Customizing the Dungeon Layout}
The layout is represented by a grid, initialized based on a specified size. This grid forms the blueprint of your dungeon. You can easily change the size of the grid to make your dungeon larger or smaller, thus altering the complexity and exploration opportunities.

\subsubsection{Strategically Placing Key Rooms}
The script includes logic for strategically placing important rooms like the Throne Room and the Entrance. You can modify this logic to change the layout dynamics, creating new challenges and narratives.

\subsubsection{Ensuring Logical Room Placement}
The function \texttt{is\_valid\_placement} checks if the placement of a room adheres to the dungeon theme and logic. Modifying this function allows you to set new rules for where specific rooms can be placed.

\subsubsection{Filling the Grid with Adventure}
After setting the key rooms, the script fills in the remaining spaces. This step is crucial for ensuring that every part of your dungeon is engaging. You can modify the filling logic to control the distribution of room types, balancing between challenge and exploration.

the default result should look like this
\begin{figure}
    \centering
    \includegraphics[width=0.5\linewidth]{bossroom.png}
    \caption{default bossroom.py result should look like this}
    \label{fig:enter-label}
\end{figure}
\subsection{Bringing Your Dungeon to Life: A Step-by-Step Example}
Let's go through an example of adding a 'Library' room to your dungeon:

\begin{lstlisting}[language=Python]
# Step 1: Add the Library to room_types
room_types["Library"] = 1

# Step 2: Assign a color to the Library in room_colors
room_colors["Library"] = "saddlebrown"

# Step 3: Modify is_valid_placement for the Library
# Add your custom logic here

# Step 4: Generate the dungeon layout
dnd_room_layout = generate_dnd_boss_room_layout_v8(6)
\end{lstlisting}



\section{Code Architecture Overview}

This section provides a detailed overview of the code architecture for two Python scripts: `bossroom.py` and `dungeonmap.py`. These scripts are designed to generate a Dungeons and Dragons (DND) dungeon layout, including a specific boss room.

\subsection{bossroom.py Overview}
\subsubsection{Purpose}
The `bossroom.py` script is designed to generate a layout for a D\&D boss room. 

\subsubsection{Key Components}
\begin{itemize}
    \item Room Types and Counts: Defines various D\&D room types and their maximum counts.
    \item Grid Initialization and Room Placement: Initializes a grid and places rooms based on specific rules.
    \item Plot Function: A function to visualize the generated room layout.
\end{itemize}

\subsection{Code Structure and Logic}
The script is structured with a clear separation between layout generation and visualization, using a mix of deterministic and random room placements.

\subsection{dungeonmap.py Overview}
\subsection{Purpose}
The `dungeonmap.py` script generates a larger dungeon map that includes the boss room.

\subsubsection{Key Components}
\begin{itemize}
    \item Grid Initialization: Initializes a grid for the dungeon map.
    \item Room Generation Functions: Functions to create and place various room types.
    \item Map Visualization: Visualizes the dungeon map with labels and descriptions.
\end{itemize}

\subsubsection{Integration with bossroom.py}
The script integrates with `bossroom.py` to incorporate the boss room layout into the larger dungeon map.

\subsection{Combined Code Architecture}
The scripts exhibit modular design, with each handling specific aspects of map generation. This modular approach aids in the readability and maintainability of the code.

\subsection{Recommendations for Future Development}
\begin{itemize}
    \item Enhance documentation and comments for clarity.
    \item Maintain modularity for ease of extending or debugging.
    \item Implement robust error handling for inputs and grid operations.
\end{itemize}


\printbibliography

\end{document}
